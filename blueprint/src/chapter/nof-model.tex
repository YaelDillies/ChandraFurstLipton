\chapter{The NOF model}

Let $G$ be a finite abelian group whose size we will denote $d$. Let $d \ge 3$ a natural number.

\begin{definition}[Forgetting a coordinate]
  \label{def:forget}
  \lean{NOF.forget}
  \leanok

  For an index $i : [d]$, we define
  \begin{align}
    \forget_i : G^d & \to G^{\{j : [d] \mid j \ne i\}} \\
    x & \mapsto j \mapsto x_j
  \end{align}
\end{definition}

\begin{definition}[NOF protocol]
  \label{def:protocol}
  \lean{NOF.Protocol}
  \leanok

  A {\bf NOF protocol} $S$ consists of maps
  \begin{align}
    \strat : & [d] \to G^{d - 1} \to \leanList \Bool \to \Bool \\
    \guess : & [d] \to G^{d - 1} \to \leanList \Bool \to G
  \end{align}
\end{definition}

We will not make $S$ part of any notation as it is usually fixed from the context.

\begin{definition}[NOF broadcast]
  \label{def:broadcast}
  \uses{def:protocol, def:forget}
  \lean{NOF.Protocol.broadcast}
  \leanok

  Given a NOF protocol $S$, the NOF broadcast on input $x : G^d$ is inductively defined by
  \begin{align}
    \broad(x) : \N & \to \leanList \Bool \\
    0 & \mapsto [] \\
    t + 1 & \mapsto \strat_{t \% d}(\forget_{t \% d}(x), \broad(x, t)) :: \broad(x, t)
  \end{align}
\end{definition}

\begin{lemma}[Length of a broadcast]
  \label{lem:length-broadcast}
  \uses{def:broadcast}
  \lean{NOF.Protocol.length_broadcast}
  \leanok
  For every NOF protocol $S$, every input $x : G^d$ and every time $t$, $\broad(x, t)$ has length $t$.
\end{lemma}
\begin{proof}
  \uses{}
  \leanok
  Induction on $t$.
\end{proof}

\begin{definition}[Valid NOF protocol]
  \label{def:valid-protocol}
  \uses{def:broadcast}
  \lean{NOF.Protocol.IsValid}
  \leanok

  Given a function $F : G^d \to G$, the NOF protocol $S$ is {\bf valid for $F$ at time $t$ on input $x$} if all participants correctly guess $F(x)$, namely if
  $$\guess_i(\forget_i(x), \broad(x, t)) = F(x)$$
  for all $i : [d]$.
\end{definition}

\begin{definition}[The trivial protocol]
  \label{def:trivial-protocol}
  \uses{def:protocol}
  \lean{NOF.Protocol.trivial}
  % \leanok

  For all $F$, we define the {\bf trivial protocol} by making participant $i$ do "Send the $t / d$-th bit of the number of participant $i + 1$" and "Compute $x_i$ from the binary representation given by participant $i - 1$, then compute $F(x)$".
\end{definition}

\begin{lemma}[The trivial protocol is valid]
  \label{lem:trivial-protocol-valid}
  \uses{def:valid-protocol, def:trivial-protocol}
  % \lean{}
  % \leanok

  For all $F$, the trivial protocol for $F$ is valid in time $d\ceil{\log_2 n}$.
\end{lemma}
\begin{proof}
  % \leanok

  Obvious.
\end{proof}

\begin{definition}[Complexity of a protocol]
  \label{def:protocol-complexity}
  \uses{def:valid-protocol}
  \lean{NOF.Protocol.complexity}
  \leanok

  The {\bf communication complexity of a NOF protocol} $S$ for $F$ is the smallest time $t$ such that the protocol is valid for $F$ at time $t$ on all inputs $x$, or $\infty$ if no such $t$ exists.
\end{definition}

\begin{definition}[Complexity of a function]
  \label{def:function-complexity}
  \uses{def:protocol-complexity}
  \lean{NOF.funComplexity}
  \leanok

  The {\bf communication complexity of a function} $F$ is the minimum of the communication complexity of $S$ when $S$ ranges over all NOF strategies.
\end{definition}

\begin{lemma}[Trivial bound on the function complexity]
  \label{lem:trivial-bound-function-complexity}
  \uses{def:function-complexity}
  % \lean{}
  % \leanok

  The communication complexity of any function $F$ is at most $d\ceil{\log_2 n}$.
\end{lemma}
\begin{proof}
  \uses{lem:trivial-protocol-valid}
  % \leanok

  The trivial protocol is a protocol valid in time $d\ceil{\log_2 n}$.
\end{proof}

\begin{definition}[Forbidden pattern]
  \label{def:forbidden-pattern}
  \uses{}
  \lean{NOF.IsForbiddenPatternWithTip, NOF.IsForbiddenPattern}
  \leanok

  We say a tuple $(a_1, \dots, a_d) : (G^d)^d$ is a {\bf forbidden pattern with tip $v : G^d$} if
  $$a_{i, j} = v_j$$
  for all $i, j$ distinct. We also simply say $(a_1, \dots, a_d)$ is a {\bf forbidden pattern} if it is a forbidden pattern with tip $v$ for some $v$.
\end{definition}

\begin{lemma}[The tip of a monochromatic forbidden pattern]
  \label{lem:mono-forbidden-pattern-tip}
  \uses{def:forbidden-pattern, def:valid-protocol}
  \lean{NOF.IsForbiddenPatternWithTip.broadcast_eq}
  \leanok

  Given $S$ a NOF protocol and a time $t$, if $(a_1, \dots, a_d)$ is a forbidden pattern with tip $v$ such that $\broad(a_i, t)$ equals some fixed broadcast history $b$ for all $i$, then $\broad(v, t) = b$ as well.
\end{lemma}
\begin{proof}
  \uses{lem:length-broadcast}
  % \leanok

  Induction on $t$. TODO: Expand
\end{proof}
